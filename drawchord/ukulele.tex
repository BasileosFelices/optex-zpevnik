%%%%%%%%%%%
% DALSI VYVOJ
% - Struny co se nehraji, kdyz se hraje akord na kytare (X)
% - Nějak speciálně označit baré akordy s posunutou kobylkou (naznačit capo?)
%%%%%%%%%%%%%%%%%%

%---Author and LINCENCE-----------------------------
% The drawchord macros were created as an extension
% of a TeX songbook by Markéta Zadražilová.
%
% This project is home at Bitbucket
% https://bitbucket.org/marketa_zadrazilova/tex/src/v2/
%
% The drawchord macros were created by 
%
%           Radek Smid
%      (rad.smid@gmail.com)
%
% and are distributed under the GNU GPLv3 licence.
%--------------------------------------------------

%---Mockup-----------------------------------------
% \chord{Ami} -- Napise Akord nad text
% \drawchord{Ami} -- Napise Akord nad text a prida jeho obrazek na okraj stranky
% \drawspecialchord{Gmi}{0}{2}{3}{2} -- Napise akord nad text a nakresli ho na okraj stranky podle definice
%

%---Typeset-----------------------------------------
\input drawchord/typesetchords

%---Draw chord--------------------------------------
\newdimen\ballradius
\ballradius = 1.6ex
\newcount\holderwidth
\holderwidth = 3
\newcount\stringwidth
\stringwidth = 1 

% Draw black dot on Position X , Position Y , Radius
% Circle macro from TeX pro pragmatiky by Petr Olšák
\def\circle#1#2{
	q
  % Circle at #1 #2
	% Draw circle with radius 0.5 at with center at 0x0
	.5 0 m
	.5 .276 .276 .5 0 .5 c
	-.276 .5 -.5 .276 -.5 0 c
	-.5 -.276 -.276 -.5 0 -.5 c
	.276 -.5 .5 -.276 .5 0 c
	% adjust size 
	5 0 0 5 0 0 cm
	% adjust position
	1 0 0 1 #1 #2 cm
	f
	S
	Q
}

\def\drawuchord#1{% It produces a simple horizontal box in the end
\hbox to 44pt{\hskip7pt%  horizontal align
\vbox to 58pt{\vskip3pt%  spring above top of the text
\hbox{\hfil\ifnum\ustringA=11? #1\else#1\fi\hfil}\vfil%
\hbox{\pdfliteral{
		q
		% -- Draw top holder --
		\the\holderwidth w
		1 j
		0 41 m
		30 41 l
		s 
		% -- Draw strings --
		\the\stringwidth w
		0 j
		00 40 m 00 -03 l S
		10 40 m 10 -03 l S
		20 40 m 20 -03 l S
		30 40 m 30 -03 l S
		% -- Draw Frets --
		00 30 m 30 30 l S
		00 20 m 30 20 l S
		00 10 m 30 10 l S
		00 00 m 30 00 l S
		% -- Draw balls --
		\ifnum\ustringA<8
		\circle{0}{\the\ustringA}
		\fi
		\ifnum\ustringB<8
		\circle{2}{\the\ustringB}
		\fi
		\ifnum\ustringC<8
		\circle{4}{\the\ustringC}
		\fi
		\ifnum\the\ustringD<8
		\circle{6}{\the\ustringD}
		\fi
		Q
	}}\vskip3pt}\hfil}}

%---Translate---------------------------------------

% Saving ukulele chord
\def\saveuchord #1#2{%
\expandafter\def\csname uchord:#1\endcsname{#2}}

% Loading ukulele chord
\def\loaduchord #1{\csname uchord:#1\endcsname}

% Inserting definitions
\input drawchord/ukuleleDICT.tex % load dictionary

% Prining based on Name of chord
\newcount\ustringA
\newcount\ustringB
\newcount\ustringC
\newcount\ustringD

% Used for checking if the chord was succesfully loaded. You cannot hold a -1 fret!
% If chord is loaded that \ustringX variables are overwritten.
\def\minusone{%
\ustringA = -1%
\ustringB = -1%
\ustringC = -1%
\ustringD = -1}

% Translate the number of fret to hold {inputted} to the Y coordinate used in pdf literal
\def\preparechord{%
\multiply\ustringA by -2\advance \ustringA by 9%
\multiply\ustringB by -2\advance \ustringB by 9%
\multiply\ustringC by -2\advance \ustringC by 9%
\multiply\ustringD by -2\advance \ustringD by 9%
}

% Draw a ukulele chord
\def\uchord #1{%
\minusone\loaduchord{#1}\preparechord\drawuchord {#1}%
}

%---Usage-----------------------------------------
% The chord is in predefined library
\def\drawchord#1{%
\chord{#1}% Vypis akord nad text
\count253 0
\insert253{{\rm\kern1em\strut\uchord{#1}}}%
}

% Chord is not in predefined library
\def\drawspecialchord#1#2#3#4#5{%
\chord{#1}% Vypis akord na text
\ustringA = #2%
\ustringB = #3%
\ustringC = #4%
\ustringD = #5%
\preparechord%
\count253 0
\insert253{{\rm\kern1em\strut\drawuchord{#1}}}%
}

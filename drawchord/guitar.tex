%---Author and LINCENCE-----------------------------
% The drawchord macros were created as an extension
% of a TeX songbook by Markéta Zadražilová.
%
% This project is home at Bitbucket
% https://bitbucket.org/marketa_zadrazilova/tex/src/v2/
%
% The drawchord macros were created by 
%
%           Radek Smid
%      (rad.smid@gmail.com)
%
% and are distributed under the GNU GPLv3 licence.
%--------------------------------------------------

%---Mockup-----------------------------------------
% \chord{Ami} -- Napise Akord nad text
% \drawchord{Ami} -- Napise Akord nad text a prida jeho obrazek na okraj stranky
% \drawspecialchord{Gmi}{0}{2}{3}{2}{0}{0} -- Napise akord nad text a nakresli ho na okraj stranky podle definice

%---Typeset-----------------------------------------
\input drawchord/typesetchords

%---Draw chord--------------------------------------
\newdimen\ballradius
\ballradius = 1.6ex
\newcount\holderwidth
\holderwidth = 3
\newcount\stringwidth
\stringwidth = 1 

% Draw black dot on Position X , Position Y , Radius
% Circle macro from TeX pro pragmatiky by Petr Olšák
\def\circle#1#2{
	q
  % Circle at #1 #2
	% Draw circle with radius 0.5 at with center at 0x0
	.5 0 m
	.5 .276 .276 .5 0 .5 c
	-.276 .5 -.5 .276 -.5 0 c
	-.5 -.276 -.276 -.5 0 -.5 c
	.276 -.5 .5 -.276 .5 0 c
	% adjust size 
	5 0 0 5 0 0 cm
	% adjust position
	1 0 0 1 #1 #2 cm
	f
	S
	Q
}

\def\drawgchord#1{% It produces a simple horizontal box in the end
\hbox to 44pt{\hskip7pt%  horizontal align
\vbox to 58pt{\vskip3pt%  spring above top of the text
\hbox{\hfil\ifnum\gstringA=11? #1\else#1\fi\hfil}\vfil%
\hbox{\pdfliteral{
		q
		% -- Draw top holder --
		\the\holderwidth w
		1 j
		0 41 m
		30 41 l
		s 
		% -- Draw strings --
		\the\stringwidth w
		0 j
		00 40 m 00 -03 l S
		06 40 m 06 -03 l S
		12 40 m 12 -03 l S
		18 40 m 18 -03 l S
		24 40 m 24 -03 l S
		30 40 m 30 -03 l S
		% -- Draw Frets --
		00 30 m 30 30 l S
		00 20 m 30 20 l S
		00 10 m 30 10 l S
		00 00 m 30 00 l S
		% -- Draw balls --
		\ifnum\gstringA<8
		\circle{0}{\the\gstringA}
		\fi
		\ifnum\gstringB<8
		\circle{1.2}{\the\gstringB}
		\fi
		\ifnum\gstringC<8
		\circle{2.4}{\the\gstringC}
		\fi
		\ifnum\the\gstringD<8
		\circle{3.6}{\the\gstringD}
		\fi
		\ifnum\the\gstringE<8
		\circle{4.8}{\the\gstringE}
		\fi
		\ifnum\the\gstringF<8
		\circle{6}{\the\gstringF}
		\fi
		Q
	}}\vskip3pt}\hfil}}

%---Translate---------------------------------------

% Saving ukulele chord
\def\savegchord #1#2{%
\expandafter\def\csname gchord:#1\endcsname{#2}}

% Loading ukulele chord
\def\loadgchord #1{\csname gchord:#1\endcsname}

% Inserting definitions
\input drawchord/guitarDICT.tex

% Prining based on Name of chord
\newcount\gstringA
\newcount\gstringB
\newcount\gstringC
\newcount\gstringD
\newcount\gstringE
\newcount\gstringF


% Used for checking if the chord was succesfully loaded. You cannot hold a -1 fret!
% If chord is loaded that \gstringX variables are overwritten.
\def\minusone{%
\gstringA = -1%
\gstringB = -1%
\gstringC = -1%
\gstringD = -1%
\gstringE = -1%
\gstringF = -1%
}

% Translate the number of fret to hold {inputted} to the Y coordinate used in pdf literal
\def\preparechord{%
\multiply\gstringA by -2\advance \gstringA by 9%
\multiply\gstringB by -2\advance \gstringB by 9%
\multiply\gstringC by -2\advance \gstringC by 9%
\multiply\gstringD by -2\advance \gstringD by 9%
\multiply\gstringE by -2\advance \gstringE by 9%
\multiply\gstringF by -2\advance \gstringF by 9%
}

% Draw a ukulele chord
\def\gchord #1{%
\minusone\loadgchord{#1}\preparechord\drawgchord {#1}%
}

%---Usage-----------------------------------------
% The chord is in predefined library
\def\drawchord#1{%
\chord{#1}% Vypis akord nad text
\count253 0
\insert253{\kern1em\strut\gchord{#1}}%
}

% Chord is not in predefined library
\def\drawspecialchord#1#2#3#4#5#6#7{%
\chord{#1}% Vypis akord na text
\gstringA = #2%
\gstringB = #3%
\gstringC = #4%
\gstringD = #5%
\gstringE = #6%
\gstringF = #7%
\preparechord%
\count253 0
\insert253{\kern1em\strut\drawgchord{#1}}%
}

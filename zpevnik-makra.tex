%---Makra pro zpěvníček-------------------------------

\input cavantga

\font\booktitf=pagk8z at32pt
\font\bookautf=pagk8z at24pt
\font\sngtitf=pagk8z scaled \magstep2
\font\sngautf=pagk8z scaled \magstep1

\parindent=0 pt


% pro debuging	
\def\boxed#1{\hbox{\vrule\vbox{\hrule\hbox{#1}\hrule}\vrule}}


% čítač slok
\newcount\verscnt


%---Pomocná makra------------------------------------
\def\newpage{\vfil \break}
\def\centerh#1{\hbox to \hsize{\hss #1 \hss}}
\def\donothing{}


%---Zpěvník------------------------------------------
\nopagenumbers							% vypnutí číslování stránek

\def\booktitle#1\par{\vbox to 0.25\vsize{}\centerh{\booktitf #1}\par\bigskip\nobreak}

\def\bookauthor#1\par{\centerh{\bookautf #1}\par\bigskip}

\def\songs{\newpage \footline={\hss\tenrm\folio\hss}}		% zapíná číslování stránek


%---Píseň--------------------------------------------
\def\beginsong{\newpage \verscnt 1\relax}			% započne číslování slok v písni

\def\title#1{\noindent {\sngtitf #1}\par
	\write\writef{\string\tocline{#1}{\the\pageno}}		% zápis informace do souboru (pro tvorbu obsahu)
	\bigskip\nobreak}

\def\author#1{\noindent {\sngautf #1}\par\bigskip\nobreak}


%---Předehra-----------------------------------------
\def\intro#1\par{\noindent {\it Intro: } #1\par\bigskip}


%---Refrén-------------------------------------------
{\catcode`\^^M=13
\gdef\beginrefrain{\begingroup \catcode`\^^M=13 \def^^M{\par\nobreak}
\noindent \bf Ref:\par\smallskip}}

\def\endrefrain{\endgroup\par\bigskip\allowbreak}

\def\refrain{\noindent{\bf Ref.}\par\bigskip\allowbreak}


%---Sloky--------------------------------------------
%TODO: Odsazení čísel slok
{\catcode`\^^M=13
\gdef\beginverse{\begingroup \catcode`\^^M=13 \def^^M{\par\nobreak}
\noindent \the\verscnt)\par\smallskip}}

\def\endverse{\endgroup\par\bigskip\allowbreak \advance\verscnt 1\relax}


%---Vysázení akordů----------------------------------
\def\chord#1{\noindent\raise 11pt \hbox to 0pt{{\it #1} \hss}}


%---Repetice-----------------------------------------
% změna catcode u znaku | aby repetice mohla být definována pomocí |:
\catcode`\|=13

{\catcode`\^^M=13
\gdef|:#1:|#2^^M{\noindent\drawl{\bi #1}\drawr\if:#2: {} \else {\bi \hskip 0.5em  #2x} \fi}}

\def\lrepsign{\pdfliteral{q
.5 w
2 10 m
0 8 l	%horní šikmá čára
0 2 l 	%kratší vertikální čára
2 0 l	%dolní šikmá čára
S
1 9 m
1 1 l	%delší přepažující vertikální čára
S
2 7 m
2 6 l	%horní bod
S
2 4 m
2 3 l	%spodní bod
S
Q}}

\def\rrepsign{\pdfliteral{q
-1 0 0 1 0 0 cm		%zrcadlení podle osy y
\lrepsign
Q}}

\newcount\lrepsignnb
\setbox0=\hbox to 3pt{\vbox to 10pt{\vss\lrepsign}\hss}
\pdfxform 0 \lrepsignnb=\pdflastxform
\def\drawl{\pdfrefxform\lrepsignnb}

\newcount\rrepsignnb
\setbox0=\hbox to 3pt{\hss \vbox to 10pt{\vss\rrepsign}}
\pdfxform 0 \rrepsignnb=\pdflastxform
\def\drawr{\pdfrefxform\rrepsignnb}


%---Obsah----------------------------------------------
% identifikátor souboru pro zápis
\newwrite\writef

% určuje jak bude vypadat zaznam v souboru s obsahem
\def\tocline#1#2{#1 \dotfill #2\par}		

% připraví soubor toc.tex pro zápis
\immediate\openout\writef=toc.tex

% vloží obsah na samostatnou stránku
\def\readtoc{\newpage \nopagenumbers
  \immediate\closeout\writef				% uzavře soubor do kterého se zapisoval obsah
  \noindent {\sngtitf Obsah:}\par\nobreak\bigskip
  \input toc						% vloží obsah souboru s obsahem
}


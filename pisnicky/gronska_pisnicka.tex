\beginsong

\title{Grónská písnička}
\author{Jaromír Nohavica}

\beginverse
D\chord{D}aleko n\chord{Emi}a severu j\chord{A7}e Grónská z\chord{D}em,
ž\chord{D}ije tam E\chord{Emi}skymačka s E\chord{D}skymákem.
|:My bychom u\chord{Emi}mrzli, jim n\chord{G}ení zim\chord{D}a,
snídají n\chord{Emi}anuky a\chord{A7} eskym\chord{D}a.:|{2}
\endverse

\beginverse
Mají se bezvadně, vyspí se moc,
půl roku trvá tam polární noc.
|:Na jaře vzbudí se a vyběhnou ven,
půl roku trvá tam polární den.:|3
\endverse

\beginverse
Když sněhu napadne nad kotníky,
hrávají s medvědy na četníky.
|:Medvědi těžko jsou k poražení,
neboť medvědy ve sněhu vidět není.:|{}
\endverse

\beginverse
Pokaždé ve středu, přesně ve dvě,
zaklepe na iglů hlavní medvěd.
|:Dobrý den, mohu dál na vteřinu?
Já nesu vám trochu ryb na svačinu.:|{}
\endverse

\beginverse
V kotlíku bublá čaj, kamna hřejí,
psi venku hlídají před zloději.
|:Smíchem se otřásá celé iglů,
medvěd jim předvádí spoustu fíglů.:|{}
\endverse

\beginverse
Tak žijou vesele na severu,
srandu si dělají z teploměrů.
|:My bychom umrzli jim není zima,
neboť jsou doma a mezi svýma.:|{}
\endverse



\beginsong

\title{Grónská písnička}
\author{Jaromír Nohavica}

\beginverse
\chord{D}Daleko \chord{Emi}na severu \chord{A7}je Grónská \chord{D}zem,
\chord{D}žije tam \chord{Emi}Eskymačka s \chord{D}Eskymákem.
|:My bychom \chord{Emi}umrzli, jim \chord{G}není zi\chord{D}ma,
snídají \chord{Emi}nanuky \chord{A7}a esky\chord{D}ma.:|{2}
\endverse

\beginverse
Mají se bezvadně, vyspí se moc,
půl roku trvá tam polární noc.
|:Na jaře vzbudí se a vyběhnou ven,
půl roku trvá tam polární den.:|3
\endverse

\beginverse
Když sněhu napadne nad kotníky,
hrávají s medvědy na četníky.
|:Medvědi těžko jsou k poražení,
neboť medvědy ve sněhu vidět není.:|
\endverse

\beginverse
Pokaždé ve středu, přesně ve dvě,
zaklepe na iglů hlavní medvěd.
|:Dobrý den, mohu dál na vteřinu?
Já nesu vám trochu ryb na svačinu.:|
\endverse

\beginverse
V kotlíku bublá čaj, kamna hřejí,
psi venku hlídají před zloději.
|:Smíchem se otřásá celé iglů,
medvěd jim předvádí spoustu fíglů.:|
\endverse

\beginverse
Tak žijou vesele na severu,
srandu si dělají z teploměrů.
|:My bychom umrzli jim není zima,
neboť jsou doma a mezi svýma.:|{}
\endverse


